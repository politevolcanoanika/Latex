\documentclass{book}
\title{ALL ABOUT LATEX}
\date{30.6.2021}
\author{Anika Samvedi}


\begin{document}
\maketitle
\newpage
\pagenumbering{roman}
Latex is a easy language, instead of using microsoft, these days iam trying to use latex, the main tags that we need to use for latex are below:
\section{Basics Of Latex}
\subsection{1.documentclass - which influence the format of the document, for example it can be a article, report, book, letter or any other thing.}
\subsection{2. title - this is usually the second line of the document, where we can give the title to the document.}
\subsection{3. date - if we want to give date to our document we can use this tag}
\subsection{4. begin  - here we start the documents, there are various sub-tags, which are used with the begin, for example- newpage, pagenumbering and many more.}
\subsection{5. end - it is used to end document.}
\paragraph{how to make a proper format of a document}
To make different sections and paragraph in a body, there are tags like, section, subsection, paragraph and subparagraph. This helps us to give a proper format to our document.
\paragraph{How to write a equation in latex}
The Below equation is written in latex, with the help of a tag, beginequation , this tag helps us to make questions for the quiz, and maths question paper obviously.
\begin{equation}
x=2y-3z+2
\end{equation}
\begin{equation}
f(x) = x^2 \\
g(x) = \frac{1}{x}\\
 F(x) = \int^a_b\frac{1}{3}x^3
\end{equation}
\paragraph{impotant info}
There is a user package for the maths functions, it is AMSMATH, with this we can use functions like matrix

\paragraph{some new things}
The new things are contentoftable tag , and user package setspace for the perfect spacing required by you, and the depth for the diffferent section to have a better effect on the sections.






\end{document}
